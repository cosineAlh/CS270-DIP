\documentclass[11pt,letterpaper]{article}
\setlength{\parindent}{0em}                  %DISTANCIA SANGRÍA
\setlength{\parskip}{0.5em}                  %DISTANCIA ENTRE PÁRRAFOS
\textwidth 6.5in
\textheight 9.in
\oddsidemargin 0in
\headheight 0in

\usepackage{fancybox}
\usepackage[utf8]{inputenc}
\usepackage{epsfig,graphicx}
\usepackage{multicol,pst-plot}
\usepackage{pstricks}
\usepackage{amsmath}
\usepackage{amsfonts}
\usepackage{amssymb}
\usepackage{eucal}
\usepackage[left=2cm,right=2cm,top=2cm,bottom=2cm]{geometry}
\usepackage{txfonts}
% \usepackage[spanish]{babel}
\usepackage[colorlinks]{hyperref}
\usepackage{cancel}
\usepackage{caption}
\usepackage{float}
\usepackage{upgreek}
\usepackage{gensymb}
\usepackage{subfigure}
\usepackage{siunitx}
\usepackage{color}
\usepackage{tikz}
\usepackage{listings}
\usepackage{minted}
\usepackage{mdframed}
\usepackage{natbib}
\usepackage{pifont}
\bibliographystyle{mnras}
\setcitestyle{aysep{","}}
\usepackage{multicol}
\renewcommand{\bibpreamble}{\begin{multicols}{2}}
\renewcommand{\bibpostamble}{\end{multicols}}
\setlength{\bibsep}{3pt}

%DEFINICIÓN DE COLORES EXTRAS

\definecolor{codegreen}{rgb}{0,0.6,0}
\definecolor{codegray}{rgb}{0.5,0.5,0.5}
\definecolor{backcolour}{rgb}{0.95,0.95,0.95}
\hypersetup{colorlinks=true,linkcolor=codegreen,citecolor=blue,filecolor=blue,urlcolor=magenta,}

%CONFIGURACIÓN DE LSTLISTINGS PARA CÓDIGOS

\lstset{ %
language=matlab,                % choose the language of the code
basicstyle=\footnotesize,       % the size of the fonts that are used for the code
numbers=left,                   % where to put the line-numbers
numberstyle=\footnotesize,      % the size of the fonts that are used for the line-numbers
stepnumber=1,                   % the step between two line-numbers. If it is 1 each line will be numbered
numbersep=5pt,                  % how far the line-numbers are from the code
backgroundcolor=\color{white},  % choose the background color. You must add \usepackage{color}
showspaces=false,               % show spaces adding particular underscores
showstringspaces=false,         % underline spaces within strings
showtabs=false,                 % show tabs within strings adding particular underscores
frame=single,                   % adds a frame around the code
tabsize=2,                      % sets default tabsize to 2 spaces
captionpos=b,                   % sets the caption-position to bottom
breaklines=true,                % sets automatic line breaking
breakatwhitespace=false,        % sets if automatic breaks should only happen at whitespace
escapeinside={\%*}{*)}          % if you want to add a comment within your code
}
\lstdefinestyle{mystyle}{
	backgroundcolor=\color{backcolour},   
	commentstyle=\color{red},
	keywordstyle=\bfseries\color{magenta},
	numberstyle=\tiny\color{codegray},
	stringstyle=\color{codegreen},
	basicstyle=\footnotesize\ttfamily,
	identifierstyle=\color{blue},
	breakatwhitespace=false,         
	breaklines=true,                 
	captionpos=b,                    
	keepspaces=true,                 
	numbers=left,                    
	numbersep=5pt,                  
	showspaces=false,                
	showstringspaces=false,
	showtabs=false,                  
	tabsize=2
}

\lstset{style=mystyle}

%CONFIGURACIÓN DE MINTED PARA CÓDIGOS

\usemintedstyle{vs}

%DEFINICIÓN DE COMANDOS EXTRAS

\pagestyle{empty}
\DeclareMathOperator{\tr}{Tr}                      %ICONO TRAZA MECANICA CUANTICA
\DeclareMathOperator{\rsol}{R_\odot}               %ICONO RADIO SOLAR
\DeclareMathOperator{\lsol}{L_\odot}               %ICONO LUMINOSIDAD SOLAR
\DeclareMathOperator{\msol}{M_\odot}               %ICONO MASA SOLAR
\DeclareMathOperator{\probabi}{Prob}               %ICONO PROBABILIDAD
\newcommand{\units}[1]{\left[ #1 \right]}          %CORCHETES PARA UNIDADES
\newcommand{\prob}[1]{\probabi\left( #1 \right)}   %OPERADOR PROBABILIDAD
\newcommand{\abs}[1]{\left|#1\right|}              %OPERADOR VALOR ABSOLUTO
\newcommand{\bra}[1]{\langle #1 |}                 %OPERADOR BRA
\newcommand{\ket}[1]{| #1 \rangle}                 %OPERADOR KET
\newcommand{\braket}[2]{\langle #1 | #2 \rangle}   %OPERADOR BRA-KET
\newcommand{\ketbra}[2]{|#1\rangle\langle#2|}      %OPERADOR KET-BRA
\newcommand{\mean}[1]{\langle #1 \rangle}          %PROMEDIO MECANICA CUANTICA
\newcommand{\eval}[3]{\left.#1\right|_{#2}^{#3}}   %COMANDO PARA EVALUAR INTEGRALES

%DEFINICIÓN DE REVISTAS CIENTÍFICAS

\newcommand\aap{A\&A}                % Astronomy and Astrophysics
\let\astap=\aap                          % alternative shortcut
\newcommand\aapr{A\&ARv}             % Astronomy and Astrophysics Review (the)
\newcommand\aaps{A\&AS}              % Astronomy and Astrophysics Supplement Series
\newcommand\actaa{Acta Astron.}      % Acta Astronomica
\newcommand\afz{Afz}                 % Astrofizika
\newcommand\aj{AJ}                   % Astronomical Journal (the)
\newcommand\ao{Appl. Opt.}           % Applied Optics
\let\applopt=\ao                         % alternative shortcut
\newcommand\aplett{Astrophys.~Lett.} % Astrophysics Letters
\newcommand\apj{ApJ}                 % Astrophysical Journal
\newcommand\apjl{ApJ}                % Astrophysical Journal, Letters
\let\apjlett=\apjl                       % alternative shortcut
\newcommand\apjs{ApJS}               % Astrophysical Journal, Supplement
\let\apjsupp=\apjs                       % alternative shortcut
% The following journal does not appear to exist! Disabled.
%\newcommand\apspr{Astrophys.~Space~Phys.~Res.} % Astrophysics Space Physics Research
\newcommand\apss{Ap\&SS}             % Astrophysics and Space Science
\newcommand\araa{ARA\&A}             % Annual Review of Astronomy and Astrophysics
\newcommand\arep{Astron. Rep.}       % Astronomy Reports
\newcommand\aspc{ASP Conf. Ser.}     % ASP Conference Series
\newcommand\azh{Azh}                 % Astronomicheskii Zhurnal
\newcommand\baas{BAAS}               % Bulletin of the American Astronomical Society
\newcommand\bac{Bull. Astron. Inst. Czechoslovakia} % Bulletin of the Astronomical Institutes of Czechoslovakia 
\newcommand\bain{Bull. Astron. Inst. Netherlands} % Bulletin Astronomical Institute of the Netherlands
\newcommand\caa{Chinese Astron. Astrophys.} % Chinese Astronomy and Astrophysics
\newcommand\cjaa{Chinese J.~Astron. Astrophys.} % Chinese Journal of Astronomy and Astrophysics
\newcommand\fcp{Fundamentals Cosmic Phys.}  % Fundamentals of Cosmic Physics
\newcommand\gca{Geochimica Cosmochimica Acta}   % Geochimica Cosmochimica Acta
\newcommand\grl{Geophys. Res. Lett.} % Geophysics Research Letters
\newcommand\iaucirc{IAU~Circ.}       % IAU Cirulars
\newcommand\icarus{Icarus}           % Icarus
\newcommand\japa{J.~Astrophys. Astron.} % Journal of Astrophysics and Astronomy
\newcommand\jcap{J.~Cosmology Astropart. Phys.} % Journal of Cosmology and Astroparticle Physics
\newcommand\jcp{J.~Chem.~Phys.}      % Journal of Chemical Physics
\newcommand\jgr{J.~Geophys.~Res.}    % Journal of Geophysics Research
\newcommand\jqsrt{J.~Quant. Spectrosc. Radiative Transfer} % Journal of Quantitiative Spectroscopy and Radiative Transfer
\newcommand\jrasc{J.~R.~Astron. Soc. Canada} % Journal of the RAS of Canada
\newcommand\memras{Mem.~RAS}         % Memoirs of the RAS
\newcommand\memsai{Mem. Soc. Astron. Italiana} % Memoire della Societa Astronomica Italiana
\newcommand\mnassa{MNASSA}           % Monthly Notes of the Astronomical Society of Southern Africa
\newcommand\mnras{MNRAS}             % Monthly Notices of the Royal Astronomical Society
\newcommand\na{New~Astron.}          % New Astronomy
\newcommand\nar{New~Astron.~Rev.}    % New Astronomy Review
\newcommand\nat{Nature}              % Nature
\newcommand\nphysa{Nuclear Phys.~A}  % Nuclear Physics A
\newcommand\pra{Phys. Rev.~A}        % Physical Review A: General Physics
\newcommand\prb{Phys. Rev.~B}        % Physical Review B: Solid State
\newcommand\prc{Phys. Rev.~C}        % Physical Review C
\newcommand\prd{Phys. Rev.~D}        % Physical Review D
\newcommand\pre{Phys. Rev.~E}        % Physical Review E
\newcommand\prl{Phys. Rev.~Lett.}    % Physical Review Letters
\newcommand\pasa{Publ. Astron. Soc. Australia}  % Publications of the Astronomical Society of Australia
\newcommand\pasp{PASP}               % Publications of the Astronomical Society of the Pacific
\newcommand\pasj{PASJ}               % Publications of the Astronomical Society of Japan
\newcommand\physrep{Phys.~Rep.}      % Physics Reports
\newcommand\physscr{Phys.~Scr.}      % Physica Scripta
\newcommand\planss{Planet. Space~Sci.} % Planetary Space Science
\newcommand\procspie{Proc.~SPIE}     % Proceedings of the Society of Photo-Optical Instrumentation Engineers
\newcommand\rmxaa{Rev. Mex. Astron. Astrofis.} % Revista Mexicana de Astronomia y Astrofisica
\newcommand\qjras{QJRAS}             % Quarterly Journal of the RAS
\newcommand\sci{Science}             % Science
\newcommand\skytel{Sky \& Telesc.}   % Sky and Telescope
\newcommand\solphys{Sol.~Phys.}      % Solar Physics
\newcommand\sovast{Soviet~Ast.}      % Soviet Astronomy (aka Astronomy Reports)
\newcommand\ssr{Space Sci. Rev.}     % Space Science Reviews
\newcommand\zap{Z.~Astrophys.}       % Zeitschrift fuer Astrophysik

%COMIENZA EL DOCUMENTO

\begin{document}

%CONFIGURACIÓN DEL ENCABEZADO

\usetikzlibrary{positioning}
\tikzset{every picture/.style={line width=0.75pt}}    
\pagestyle{plain}
\begin{flushleft}
Digital Image Processing\\
School of Information Science and Technology\\
\underline{ShanghaiTech University}
\end{flushleft}

% \begin{flushright}\vspace{-5mm}
% \includegraphics[height=1.5cm]{shanghaitech.jpg}
% \end{flushright}
 
\begin{center}\vspace{-1cm}
\textbf{\large Assignment 1}\\  
Due time: 23:59,	March 15th,	2023\\
\end{center}
\rule{\linewidth}{0.1mm}

\section{Notes}
This homework has \textbf{100 points} in total. \par
Please submit your homework to blackboard with a zip file named as \textcolor[rgb]{1,0,0}{\textbf{DIP2023\_ID\_Name\_hw1.zip}}. The zip file should contain three things: \textcolor[rgb]{1,0,0}{\textbf{a folder named 'codes' storing your codes}},  \textcolor[rgb]{1,0,0}{\textbf{a folder named 'images' storing the original images}}, and \textcolor[rgb]{1,0,0}{\textbf{your report named as report\_ID\_Name\_hw1.pdf}}. The names of your codes should look like \textcolor[rgb]{1,0,0}{\textbf{'p1a.m'}} (for (a) part of Problem $1$), so that we can easily match your answer to the question. \textcolor[rgb]{1,0,0}{Make sure all paths in your codes are relative path and we can get the result directly after running the code}. Please answer in \textcolor[rgb]{1,0,0}{English}. \par

Please complete all the coding assignments using \textcolor[rgb]{1,0,0}{MATLAB}. All core codes are required to be implemented \textcolor[rgb]{1,0,0}{by yourself} (without using relevant built-in functions). Make sure your results in the report are the same with the results of your codes. Please explain with notes at least at the key steps of your code.

\section{Policy on plagiarism}
This is an individual homework. You can discuss the ideas and algorithms, but you can neither read, modify, and submit the codes of other students, nor allow other students to read, modify, and submit your codes. Do not directly copy ready-made or automatically generated codes, or your score will be seriously affected. We will check plagiarism using automated tools and any violations will result in a zero score for this assignment. 

\clearpage

\section{Problem sets}

\subsection*{Problem 1 (30 pts)}
$$
\begin{bmatrix}
6 & 1 & 2 & 1 & (2)\\ 
2 & 3 & 5 & 3 & 8\\ 
1 & 0 & 1 & 2 & 3\\ 
3 & 2 & 4 & 5 & 2\\ 
(1) & 5 & 3 & 4 & 0
\end{bmatrix}
$$
\begin{itemize}
\item[(a)] Calculate the $D_4,D_8,D_m$ distance between the pair of points(framed in parentheses) in the matrix above, mark the shortest 4-,8-,m-path respectively. Show the results in your report. (V=1,2,3) (10 pts)
\item[(b)] An affine transformation of coordinates is given by
$$
\begin{bmatrix}
x'
\\ 
y'
\\ 
1
\end{bmatrix}
=A\begin{bmatrix}
x\\ y\\ 1
\end{bmatrix}
=\begin{bmatrix}
a_{11} & a_{12} & a_{13}\\ 
a_{21} & a_{22} & a_{23}\\ 
0 & 0 & 1
\end{bmatrix}\begin{bmatrix}
x\\y 
\\ 1
\end{bmatrix}
$$
Please perform the following transformation on \textbf{jetplane.tif}, write out the affine transformation matrix, then show the results:

(1) Scaling and shearing. Set x-scaling factor to 2 and y-scaling factor to 3, then shear the image horizontally 4
units. (5 pts)

(2) Translation and Rotation. Move 2 units to the left and move down 5 units, then rotate 45 degrees clockwise
(about the origin). (5 pts)

(\textcolor[rgb]{1,0,0}{Hint:} The main focus of this problem is to solve the affine transformation matrix, and using built-in functions like imwarp() to obtain the transformed image is allowed.)
\item[(c)] Perform bit-plane slicing on \textbf{lena\_gray\_256.tif} to get bit plane 1-8 and show the results in your report. Which kind of bit-plane usually contains more effective information, low bit or high bit? Why? (10 pts)
\end{itemize}

\textbf{Solution:}

\clearpage

\subsection*{Problem 2 (30 pts)}
\begin{itemize}
\item[(a)] Compute the histogram of \textbf{einstein\_low\_contrast.tif}. Perform histogram equalization (HE) on it, show the result. Show the histogram of the processed image, too. Why can HE enhance the contrast? (10 pts)

\item[(b)] Match the histgram of \textbf{lena\_color.tif} to \textbf{peppers\_color.tif}. Show the result in your report. (8 pts)

\item[(c)] Perform contrast limited adptive histogram equalization(CLAHE) on \textbf{man\_in\_house.png}. Contrast the result to HE method. Show the results in your report. Explain why CLAHE has a better effect.(Reference: http://cas.xav.free.fr/Graphics\%20Gems\%204\%20-\%20Paul\%20S.\%20Heckbert.pdf, 
 page474-485) (12 pts)

(\textcolor[rgb]{1,0,0}{Hint:} If you can effectively eliminate the checkerboard effect and balance the efficiency of the algorithm, you will get a bonus. Built-in functions like hist(), histogram(), histeq() and adapthisteq() are not allowed.)
\end{itemize}

\textbf{Solution:}
\clearpage

\subsection*{Problem 3 (40 pts)}

\begin{itemize}
    \item[(a)] Implement full convolution and cropped convolution of the given matrices, show the results in your report. (8 pts)\\

$$
 \begin{bmatrix}
6 & 4 & -1 & 0 & 1\\ 
1 & -3 & -4 & 3 & 2\\ 
0 & 3 & 5 & -2 & 1\\ 
9 & -1 & -3 & 4 & 5\\ 
-2 & -5 & 2 & 3 & 0
\end{bmatrix}\quad
\begin{bmatrix}
1 & 2 & 4\\ 
-3 & -1 & 1\\ 
5 & 2 & -1
\end{bmatrix}
$$

(\textcolor[rgb]{1,0,0}{Hint:} Pay attention to the difference between convolution and correlation.)

    \item[(b)] Describe the noise type on \textbf{circuitboard-a.tif}, \textbf{circuitboard-b.tif}, \textbf{circuitboard-c.tif} and \textbf{circuitboard-d.tif}, then choose the best filter you think for each of them to reduce the noise. Show the results in your report. (12 pts)

    (\textcolor[rgb]{1,0,0}{Hint:} Choose the best kernel size you think.)

    \item[(c)] Filter the image \textbf{house.tif} in both x direction and y direction with 3*3 Sobel mask and Laplacian mask. Show the results in your report. (8 pts)

    \item[(d)] Perform image sharpening on \textbf{house.tif} using LoG filter(with the $\sigma^2 = 1$ and another two values you choose yourself) and unsharpen mask method(choose the best $k$ you think). Show the results in your report. (12 pts)
\end{itemize}

\textbf{Solution:}

\end{document}
